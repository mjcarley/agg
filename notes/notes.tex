\documentclass[a4paper,12pt]{article}

\usepackage{times}
\usepackage{amsmath}
\usepackage{graphicx}

\newcommand{\sgn}{\mathrm{sgn}}

\begin{document}

Closed curves (``shapes'') used to define cross-sections are
parameterized using $x$, $-1\leq x \leq 1$ with $x<0$ defining the
lower portion of the curve and $x>0$ the upper. The variable runs
continuously from the trailing edge along the lower surface to the
leading edge and back to the trailing edge. Coordinates are calculated
as
\begin{align}
  \label{equ:curves}
  y_{U} &= x^{n_{1}}(1-x)^{n_{2}}\sum_{i=0}^{n} s_{i}^{(U)}S_{n,i}(x),\\
  y_{L} &= -|x|^{n_{1}}(1-|x|)^{n_{2}}\sum_{i=0}^{n} s_{i}^{(L)}S_{n,i}(|x|).
\end{align}

``Aerofoils'' are defined similarly but in terms of the thickness
distribution, the trailing edge thickness, and the camber line. 

\begin{table}
  \centering
  \begin{tabular}{lll}
    \verb+AGG_TRANSFORM_NULL+  & $\mathbf{x}$& \\
    \verb+AGG_TRANSFORM_REVERSE+  & $[-x,\, y]$ & reverse direction of curve\\
    \verb+AGG_TRANSFORM_PLANE_ROTATE+  &
    $
    \displaystyle
    \mathbf{x}_{0} +
    (\mathbf{x}-\mathbf{x}_{0})\mathbf{R}^{T}
    $
    &
    rotate about $\mathbf{x}_{0}$\\
    \verb+AGG_TRANSFORM_SHRINK+  &
    $\displaystyle\mathbf{x}_{0} + s(\mathbf{x}-\mathbf{x}_{0})$
    & scale about $\mathbf{x}_{0}$\\
    \verb+AGG_TRANSFORM_SHIFT+  & $\mathbf{x} + \Delta\mathbf{x}$&
    3D translation by $\Delta \mathbf{x}$\\
    \verb+AGG_TRANSFORM_SCALE+  & $s\textbf{x}$& global scale in plane
    \\
  \end{tabular}
  \caption{Transforms in order of application to a shape}
  \label{tab:transforms}
\end{table}

Definition of a sphere of unit diameter centred at $[0,0,1/2]$:
\begin{subequations}
  \label{equ:sphere}
  \begin{align}
    x &= 2u^{1/2}(1-u)^{1/2}(|v|-1/2);\\
    y &= \sgn(v)2u^{1/2}(1-u)^{1/2}|v|^{1/2}(1-|v|)^{1/2};\\
    z &= u,\\
    0\leq u\leq 1,\,-1\leq v\leq 1.\nonumber
  \end{align}
\end{subequations}

To generate using AGG:
\begin{verbatim}
global {
# syntax: <identifier> = "<expression>" (quotes required)
# diameter of sphere
  D = 1.0
  n1 = 0.5
  n2 = 0.5
}

distribution("sphere", 0, 1, 24, 9, 17, "cosine", "cosine", "cosine")
{
  shape("ellipse", -1, 1, n1, n2) 
  transform("shrink", 0.5, 0.0, "sqrt(t*(1-t))")
  transform("scale", "2*D")
  transform("shift", "-D", 0.0, "D*t")
}
\end{verbatim}

\end{document}

%%% Local Variables:
%%% mode: latex
%%% TeX-master: t
%%% End:
